\begin{frame}
	\frametitle{Obtaining Moltres}
               \begin{figure}[t]
                \includegraphics[height=0.15\textwidth]{./images/clone_moltres.png}
               \end{figure}   
\end{frame}



\begin{frame}
  \frametitle{Moltres (coupling in MOOSE)}
   \begin{block}{Moltres principal concept \cite{lindsay_introduction_2018}}
     \begin{itemize}
        \item Moltres is built on top of the Multi-physics Object-Oriented
Simulation Environment (MOOSE)
		\item MOOSE interfaces with libMesh to discretize simulation volume
into finite elements
		\item Provides interface for coding residuals that correspond to weak
form of governing PDEs; also interface for coding Jacobians $\Rightarrow$
more accurate Jacobians $\Rightarrow$ more efficient convergence
		\item Residuals and Jacobians send to PetSc which handles solution of resulting non-linear system of algebraic equations
     \end{itemize}
    \end{block}
               \begin{figure}[t]
                \vspace*{-0.15in}
                \includegraphics[height=0.24\textwidth]{./images/moltres-moose-diag.png}
                \vspace*{-0.1in}
                \caption{Moltres principal scheme}
               \end{figure}   
\end{frame}

\begin{frame}
	\frametitle{Intro to Moltres}
    	\begin{itemize}
			\item Liquid-fueled, molten salt reactors
			\item Multi-group diffusion (arbitrary number of groups)
			\item Advective movement of delayed neutron precursors
			\item Reynolds-averaged Navier-Stokes thermal hydraulics
			\item 2D axisymmetric
			\item 3D unstructured or structured
	    \end{itemize}
\end{frame}


\begin{frame}
  \frametitle{Moltres Kernels}
    \begin{columns}
    \column[t]{6cm}
               \begin{figure}[t]
                  \vspace*{-0.15in}
                 \hspace*{-0.35in}
                \includegraphics[height=1.15\textwidth]{./images/kernels4.png}
               \end{figure}   

    \column[t]{6cm}
	\begin{itemize}
		\item Kernels are individual pieces of governing equations
		\item Modular (i.e. ``Diffusion" kernel could be used equally well to describe conduction or viscous shear)
	\end{itemize}
    \end{columns}  
  
\end{frame}

\begin{frame}
  \frametitle{Moltres Kernels (2)}
               \begin{figure}[t]
               \vspace*{-0.1in}
                 \hspace*{-0.35in}
                \includegraphics[height=0.6\textwidth]{./images/kernels11.png}
               \end{figure}   
\end{frame}


\begin{frame}
  \frametitle{Governing Equations}
      \begin{block}{Time-dependent multi-group diffusion}
              \begin{figure}[t]
               \hspace*{-0.25in}
                \includegraphics[height=0.55\textwidth]{./images/diffusion.png}
               \end{figure}   
      \end{block}
\end{frame}

\begin{frame}
  \frametitle{Governing Equations (2)}
     \begin{block}{Delayed neutron precursors}
               \begin{figure}[t]
                 \vspace*{-0.05in}
                \includegraphics[height=0.07\textwidth]{./images/delayed_neutrons.png}
               \end{figure}   
     \end{block}
     
     \begin{block}{Heat conduction-convection with fission source in fuel}
              \begin{figure}[t]
                 \vspace*{-0.05in}
                \includegraphics[height=0.21\textwidth]{./images/fuel_temp.png}
               \end{figure}        
	\end{block}
	
     \begin{block}{Heat conduction with option for irradiation source in moderator}
              \begin{figure}[t]
                 \vspace*{-0.05in}
                 \includegraphics[height=0.17\textwidth]{./images/moder_temp.png}
               \end{figure}        
	\end{block}
\end{frame}

\begin{frame}
  \frametitle{Moltres \gls{MSRE} Simulation}
              \begin{figure}[t]
               \hspace*{-0.25in}
                \includegraphics[height=0.65\textwidth]{./images/msre.png}
               \end{figure}   
\end{frame}

\begin{frame}
  \frametitle{Moltres MSRE Simulation: Input Data}
     \begin{block}{Main input parameters \cite{lindsay_introduction_2018}}
     	\begin{itemize}
     		\item 22.5\% volume fraction fuel
     		\item Group constants generated with SERPENT or NEWT (3D or 2D)
     	\end{itemize}
              \begin{figure}[t]
                 \vspace*{-0.15in}
                 \hspace*{-2.0in}
                \includegraphics[height=0.2\textwidth]{./images/moltres-composition.png}
               \end{figure}     
               \begin{figure}[t]
                 \vspace*{-0.25in}
                \includegraphics[height=0.33\textwidth]{./images/moltres-input.png}
               \end{figure}   

	\end{block}
	
\end{frame}