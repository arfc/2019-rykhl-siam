\begin{frame}
  \frametitle{Conclusions}
		\begin{block}{Moltres}
		\begin{itemize}
		\item New tool \textbf{Moltres} was developed for modeling coupled physics in novel molten salt reactors
		\item 2D-axisymmetric and 3D multiphysics models are presented
		\item \textbf{Moltres} demonstrated strong parallel scaling (up to 384 physical cores) but further optimization required
        \end{itemize}
        \end{block}
        
\end{frame}

\begin{frame}
  \frametitle{Future work}       
              \begin{block}{Future research effort}
               \begin{enumerate}
                \item Validation with other multiphysics codes for various \gls{MSR} designs
                \item Include two-phase flow capabilities (gaseous fission products appear in the liquid salt, hence, salt cannot be considered incompressible)
                \item Flow vortices and similar thermal hydraulic phenomena representation
                \item Add realistic natural circulation (better than Boussinesq approximation)
                \item Incorporate compressibility of fuel salt
                \item Implement higher fidelity methods for neutron transport 
                \item Realistic thermal hydraulic will enable more realistic precursor drift
               \end{enumerate}
               \end{block}
\end{frame}

\begin{frame}
  \frametitle{Future work}       
              \begin{block}{Improving Moltres TH capabilities with high fidelity Nek5000 CFD framework \cite{obabko_verification_2015}}
               \begin{enumerate}
                \item Investigate vortex formation with single channel and partial core models in Nek5000
                \item Based on results decide which method to use instead of Reynolds-averaged Navier-Stokes
                \item Generate Reduced Order Model based on produced data
                \item The model will be used to establish functional kernel forms for laminar-turbulent transitional flow behavior predictive of vortices in Moltres
               \end{enumerate}
               \end{block}
\end{frame}